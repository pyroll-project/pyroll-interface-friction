%! suppress = MissingImport
%! Author = Christoph Renzing
%! Date = 10.03.22

% Preamble
\documentclass[11pt]{PyRollDocs}

\addbibresource{refs.bib}

\newmintinline[py]{python}{}

% Document
\begin{document}

    \title{The Interface Friction PyRolL Plugin}
    \author{Christoph Renzing}
    \date{\today}

    \maketitle

    The Interface Friction plugin serves as a base package for other plugins which include friction to calculate relevant process values.
    It is mainly intended for calculations regarding groove filling, spread, stresses as well as forces and torques witch result from the process.
    For further descriptions regarding friction in metal forming processes see \textcite{Black1993} or \textcite{Wilson1987}.


    \section{Model Approach}\label{sec:model-approach}

    The interface friction model introduces two variables as hooks.
    Namely, these variables are the friction coefficient ($\mu$) according to Coulomb's sliding friction model as well as a friction factor ($m$) which is used for sticking friction.
    Both values can be calculated using the respective other through a equation given by \textcite{Wanheim1974}.

    \begin{equation}
        \mu = \frac{m}{1 + \frac{\pi}{2} + \arccos\left( m \right) + \sqrt{1 - m ^2}}
        \label{eq:friction-coefficient-factor}
    \end{equation}


    \section{Usage Instructions and Implementation Details}\label{sec:usage-instructions}

    Packages residing on friction shall depend on this plugin to create a common interface.
    In the following, the hooks defined by this plugin shall be described.

    \begin{description}
        \item[\py/RollPass.coulomb_friction_coefficient/] Friction coefficient of Coulombs friction model used for sliding friction.
        \item[\py/RollPass.friction_factor/] Friction factor for sticking friction.
    \end{description}

    \printbibliography

\end{document}